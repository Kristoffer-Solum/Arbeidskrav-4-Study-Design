% Options for packages loaded elsewhere
\PassOptionsToPackage{unicode}{hyperref}
\PassOptionsToPackage{hyphens}{url}
%
\documentclass[
]{article}
\usepackage{amsmath,amssymb}
\usepackage{lmodern}
\usepackage{ifxetex,ifluatex}
\ifnum 0\ifxetex 1\fi\ifluatex 1\fi=0 % if pdftex
  \usepackage[T1]{fontenc}
  \usepackage[utf8]{inputenc}
  \usepackage{textcomp} % provide euro and other symbols
\else % if luatex or xetex
  \usepackage{unicode-math}
  \defaultfontfeatures{Scale=MatchLowercase}
  \defaultfontfeatures[\rmfamily]{Ligatures=TeX,Scale=1}
\fi
% Use upquote if available, for straight quotes in verbatim environments
\IfFileExists{upquote.sty}{\usepackage{upquote}}{}
\IfFileExists{microtype.sty}{% use microtype if available
  \usepackage[]{microtype}
  \UseMicrotypeSet[protrusion]{basicmath} % disable protrusion for tt fonts
}{}
\makeatletter
\@ifundefined{KOMAClassName}{% if non-KOMA class
  \IfFileExists{parskip.sty}{%
    \usepackage{parskip}
  }{% else
    \setlength{\parindent}{0pt}
    \setlength{\parskip}{6pt plus 2pt minus 1pt}}
}{% if KOMA class
  \KOMAoptions{parskip=half}}
\makeatother
\usepackage{xcolor}
\IfFileExists{xurl.sty}{\usepackage{xurl}}{} % add URL line breaks if available
\IfFileExists{bookmark.sty}{\usepackage{bookmark}}{\usepackage{hyperref}}
\hypersetup{
  pdftitle={Study Design},
  pdfauthor={Kristoffer Solum},
  hidelinks,
  pdfcreator={LaTeX via pandoc}}
\urlstyle{same} % disable monospaced font for URLs
\usepackage[margin=1in]{geometry}
\usepackage{graphicx}
\makeatletter
\def\maxwidth{\ifdim\Gin@nat@width>\linewidth\linewidth\else\Gin@nat@width\fi}
\def\maxheight{\ifdim\Gin@nat@height>\textheight\textheight\else\Gin@nat@height\fi}
\makeatother
% Scale images if necessary, so that they will not overflow the page
% margins by default, and it is still possible to overwrite the defaults
% using explicit options in \includegraphics[width, height, ...]{}
\setkeys{Gin}{width=\maxwidth,height=\maxheight,keepaspectratio}
% Set default figure placement to htbp
\makeatletter
\def\fps@figure{htbp}
\makeatother
\setlength{\emergencystretch}{3em} % prevent overfull lines
\providecommand{\tightlist}{%
  \setlength{\itemsep}{0pt}\setlength{\parskip}{0pt}}
\setcounter{secnumdepth}{-\maxdimen} % remove section numbering
\ifluatex
  \usepackage{selnolig}  % disable illegal ligatures
\fi
\newlength{\cslhangindent}
\setlength{\cslhangindent}{1.5em}
\newlength{\csllabelwidth}
\setlength{\csllabelwidth}{3em}
\newenvironment{CSLReferences}[2] % #1 hanging-ident, #2 entry spacing
 {% don't indent paragraphs
  \setlength{\parindent}{0pt}
  % turn on hanging indent if param 1 is 1
  \ifodd #1 \everypar{\setlength{\hangindent}{\cslhangindent}}\ignorespaces\fi
  % set entry spacing
  \ifnum #2 > 0
  \setlength{\parskip}{#2\baselineskip}
  \fi
 }%
 {}
\usepackage{calc}
\newcommand{\CSLBlock}[1]{#1\hfill\break}
\newcommand{\CSLLeftMargin}[1]{\parbox[t]{\csllabelwidth}{#1}}
\newcommand{\CSLRightInline}[1]{\parbox[t]{\linewidth - \csllabelwidth}{#1}\break}
\newcommand{\CSLIndent}[1]{\hspace{\cslhangindent}#1}

\title{Study Design}
\author{Kristoffer Solum}
\date{12 11 2021}

\begin{document}
\maketitle

\hypertarget{study-design}{%
\section{Study Design}\label{study-design}}

\hypertarget{inntroduksjon}{%
\subsection{Inntroduksjon}\label{inntroduksjon}}

\hypertarget{metode}{%
\subsection{Metode}\label{metode}}

Formålet i alle studiene handler i hovedsak om å undersøke sammenhengen
mellom intensitet ved kortintervaller og hor mye tid det gir over en
gitt \% av vo2 maks. studien til (\textbf{thevenet2007og?}) Wakefield
and Glaister (2009) svarergodt på hverandre hvor begge sammenligner
effekten av treningsintensitet på tid over 95\% av VO2 maks, Thevenet et
al. (2007a) har en hypotese som tilsier at en økning i hastigheten fra
100\% av farten ved vo2maks til 115\% av farten ved vo2maks under
intervallene ville også gi en økning i tiden over 90- og 95\% av
vo2maks. Mens Wakefield and Glaister (2009) har en motsatt hypotese om
at tid på eller over 95\% av vo2maks ville være lenger ved løp på 105\%
av farten ved vo2 maks enn ved 115\% av farten ved vo2 maks. Studiene
til V. L. Billat et al. (2000) og V. Billat et al. (2001) til forskjell
intervaller hvor de ser på hvordan man kan fremkalle mest tid ved 100\%
av vo2maks, deriblant hvilken intensitet og intervallform som er best
for dette. Mens Dupont et al. (2002) passer bra inn med alle de andre
artiklene over og ser på både tid ved vo2maks og tiden over 90\% av
vo2maks ved kortintervaller 15/15.

Alle studiene presenterte en hypotese. Hypotesen de kommer med er
begrunnet av tidligere forskingsdata. Som et eksempel begrunner
Wakefield and Glaister (2009) sin hypotese ved å referere til data i
studien til Thevenet et al. (2007a). Gjennom å presentere denne dataen
fra tidligere forskning skapes det en logisk linje gjennom
introduksjonen til hypotesen. Dette blir gjort på en spesielt god måte i
artikkelen til Thevenet et al. (2007a) hvor artikkelen starter bredt med
å fortelle om vo2maks, og at det er ansett som en viktig faktor innenfor
langdistanseløping. Dette blir bygget opp ved å referere til tidligere
artikler som har kommet frem til dette. Videre blir det beskrevet
treningsmetoder for å forbedre vo2maks, hvor mye tid på en høy \% av
vo2maks blir beskrevet som en viktig faktor spesielt tid over 90- og
95\% av vo2maks. Etter dette beveger introduksjonen i artikkelen seg
over på å fortelle om intervalltrening og spesielt kortintervaller og
dens effekt på å kunne arbeide på en høy \% av vo2 maks. artikkelen
forteller videre om andre forskingsartikler som har sett på
kortintervaller og tid over 90 og 95\% men avdekker samtidig
kunnskapshull i disse artiklene. Dette gjør at Thevenet et al. (2007a)
formulerer en hypotese basert på informasjonen om vo2maks og
kortintervaller som skal forsøke å tette disse kunnskapshullene som ble
beskrevet i introduksjonen.\\
Dermed funker denne introduksjonen som en trakt som starter med å
fortelle om et tema bredt, og smaler seg etter hvert inn på mindre
temaer og spesifikke problemer den ønsker å svare på, samtidig som den
har en rød trå eller logisk linje gjennom introduksjonen og frem til
hypotesen. De andre artiklene følger i likhet denne samme strukturen i
sin innledning som leder til en hypotese.

Siden alle studiene på forhånd hadde spesifisert en primær hypotese
stemmer dette godt med designet i en clinical tril. En god regel for
spessielt clinical trils, er å etablere flere hypoteser på forhånd som
gir mening, men å spesifisere en hypotese som den primære, denne må
kunne bli statistisk testet uten argument om at hyoptesen må endres
(Hulley, 2013 s53). mer viktig vil en primær hypotese hjelpe studien å
fokusere på dens hovedoppgaver og bidra til en ren base for
kalkulasjoner av utvalgsstørrelsen (Hulley, 2013 s53).

Alle studiene hadde en god beskrivelse av forsøkspersonene som var med i
testen. De viste alle frem med standardavvik antall deltakere, kjønn,
alder, høyde og vekt. Alle studiene fortalte også at alle deltakere var
godt trente personer, at de hadde meldt seg som frivillige og at de var
fullt informert over test prosedyren og hadde underskrevet på dette,
foresatte underskrev for deltakere under 18 år. Alderen på deltakerne i
de forskjellige studiene varierte mye hvor Thevenet et al. (2007b) hadde
den yngste snittalderen på 16,1± 1 mens V. Billat et al. (2001) hadde
den eldste snittalderen på 51±6. Ingen av disse to studiene forsvarte
hvorfor de valgte å teste disse aldersgruppene på henholdsvis yngre og
litt eldre utøvere.\\
I testen til Thevenet et al. (2007a) ble det forklart at deltakeren alle
var rekruttert fra samme treningsklubb, og i studiene til både Dupont et
al. (2002) og Wakefield and Glaister (2009) var deltakeren
idrettsstudenter. I de to resterende artiklene var det ikke oppgitt hvor
deltakerne var rekruttert fra V. Billat et al. (2001) og V. L. Billat et
al. (2000) . Alle studiene hadde forholdsvis få deltakere med mellom 7-9
personer, og ingen av studien hadde noen forklaring på hvorfor de hadde
det antallet deltakere. Dette er litt underlig ettersom en studie med
flere deltakere ville gitt en større statistisk styrke, fordi det da vil
være fler observasjoner som påvirker gjennomsnittet.\\
I alle studiene ble det brukt p-verdier som en teststyrke hvor
signifikantnivået var P\textless0,05. ingen av studiene brukte andre
tester som t- test eller effektstørrelse som kunne ha bidratt til å
styrke resultatet. Hvordan testene ble gjennomført var svært
forskjellige Wakefield and Glaister (2009) gjennomførte testen på mølle
med 1\% stigning mens resten av testene ble gjennomført på løpe bane.
Alle studiene hadde intervaller som en av testene, her var alle
kortintervaller med mellom 15- 30sek arbeidstid. Før intervallene
gjennomførte alle studiene vo2maks test, som ble bestemmende for farten
de skulle løpe på under intervallene. Dupont et al. (2002) hadde den
raskeste hastigheten med 15/15s intervaller på 140\% av farten ved
vo2maks, mens Thevenet et al. (2007a) hadde intervaller med tregest
hastighet 30/30sek med 100\% av farten ved vo2maks. Under alle testene
ble gassutveksling målt, dette innebar variabler som ventilasjon,
pustefrekvens, oksygen, og co2. I tillegg målte alle studiene laktat
mens V. Billat et al. (2001) og V. L. Billat et al. (2000) var de eneste
som ikke målte hjertefrekvens. Av disse målingene var oksygenmålingen
den eneste variabelen som direkte svarte på hypotesen i alle studiene.

Alle studiene adresserte hypotesen deres i konklusjonen, og diskuterte
rundt resultatene de fikk. I studiene til V. L. Billat et al. (2000) og
Thevenet et al. (2007a) foreslo studiene at ytterligere forskning
krevdes for å kunne fastslå resultatet helt. Dette var fordi forskingen
deres ikke så på langtidsvirkningen av intervallene og det mente de var
nødvendig for å kunne fastslå om den typen intervall som prestere best i
deres studie faktisk ville være den beste for utvikling av
kardiovaskulære systemer over tid. I tillegg ~var det interessant å se
at alle studiene antatt Dupont et al. (2002) kun snakket om sine egne
resultater i konklusjonen. Mens Dupont et al. (2002) refererte til andre
studier i sin egen konklusjon.

\hypertarget{refferanser}{%
\subsection{Refferanser}\label{refferanser}}

Hulley, S. B. (2013). \emph{Designing clinical research} (4th ed.).

(V. L. Billat et al. 2000; V. Billat et al. 2001; Dupont et al. 2002;
Thevenet et al. 2007a; Wakefield and Glaister 2009)

\hypertarget{refs}{}
\begin{CSLReferences}{1}{0}
\leavevmode\hypertarget{ref-billat2001}{}%
Billat, V., J. Slawinksi, V. Bocquet, P. Chassaing, A. Demarle, and J.
Koralsztein. 2001. {``Very Short (15 s - 15 s) Interval-Training Around
the Critical Velocity Allows Middle-Aged Runners to Maintain V{{}}O
{\textsubscript{2}} Max for 14 Minutes.''} \emph{International Journal
of Sports Medicine} 22 (03): 201--8.
\url{https://doi.org/10.1055/s-2001-16389}.

\leavevmode\hypertarget{ref-billat2000}{}%
Billat, Véronique L., Jean Slawinski, Valery Bocquet, Alexandre Demarle,
Laurent Lafitte, Patrick Chassaing, and Jean-Pierre Koralsztein. 2000.
{``Intermittent Runs at the Velocity Associated with Maximal Oxygen
Uptake Enables Subjects to Remain at Maximal Oxygen Uptake for a Longer
Time Than Intense but Submaximal Runs.''} \emph{European Journal of
Applied Physiology} 81 (3): 188--96.
\url{https://doi.org/10.1007/s004210050029}.

\leavevmode\hypertarget{ref-dupont2002}{}%
Dupont, Grégory, Nicolas Blondel, Ghislaine Lensel, and Serge Berthoin.
2002. {``Critical Velocity and Time Spent at a High Level of for Short
Intermittent Runs at Supramaximal Velocities.''} \emph{Canadian Journal
of Applied Physiology} 27 (2): 103--15.
\url{https://doi.org/10.1139/h02-008}.

\leavevmode\hypertarget{ref-thevenet2007a}{}%
Thevenet, Delphine, Magaly Tardieu, Hassane Zouhal, Christophe Jacob,
Ben Abderraouf Abderrahman, and Jacques Prioux. 2007b. {``Influence of
Exercise Intensity on Time Spent at High Percentage of Maximal Oxygen
Uptake During an Intermittent Session in Young Endurance-Trained
Athletes.''} \emph{European Journal of Applied Physiology} 102 (1):
19--26. \url{https://doi.org/10.1007/s00421-007-0540-6}.

\leavevmode\hypertarget{ref-thevenet2007}{}%
---------. 2007a. {``Influence of Exercise Intensity on Time Spent at
High Percentage of Maximal Oxygen Uptake During an Intermittent Session
in Young Endurance-Trained Athletes.''} \emph{European Journal of
Applied Physiology} 102 (1): 19--26.
\url{https://doi.org/10.1007/s00421-007-0540-6}.

\leavevmode\hypertarget{ref-wakefield2009}{}%
Wakefield, Benjamin R, and Mark Glaister. 2009. {``Influence of
Work-Interval Intensity and Duration on Time Spent at a High Percentage
of
JOURNAL/Jscr/04.03/00124278-200912000-00017/Ov0312{\_}5/v/2021-02-09t093646z/r/Image-Png
O2max During Intermittent Supramaximal Exercise.''} \emph{Journal of
Strength and Conditioning Research} 23 (9): 2548--54.
\url{https://doi.org/10.1519/JSC.0b013e3181bc19b1}.

\end{CSLReferences}

\end{document}
